\subsubsection{Confusion matrix}
Confusion matrix is a technique for feature selection. It allows for mapping of different values from the dataset to the table for visualisation \parencite{web:PrecisionAndRecallInPython}. Confusion matrix is used (most often) in classification, attribute selection and k-nearest neighbors \parencite{visa2011confusion}
\newline
There are four elements confusion matrix represents:
\begin{itemize}
  \item True Positives (TP), 
  \item True Negatives (TN),  
  \item False Positives (FP), 
  \item False Negatives (FN).
\end{itemize}
Confusion matrix can represent these three (main) calculations:
\begin{itemize}
  \item precision,
  \item recall,
  \item f1 score.
\end{itemize}

In this case, we are using module confusion matrix from SkLearn \parencite{web:ConfusionMatrix}. The data that is passed in the function of confusion matrix is y{\_}true (confirmed true cases) and y{\_}pred (predicted true cases). Function is then returning four elements (TP, TN, FP and FN).