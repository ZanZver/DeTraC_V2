\subsubsection{Test case 4}
This test is about investing an effect of augmented data. In the Test case 3, we have had only 100 images which is not a lot. To combat the small dataset size, we can use data augmentation. So, in this dataset, we have used 100 images per class and augmented them. This gave us 730 images per class which in theory should result in the better accuracy.
\newline
Results that are represented in the graph for composed dataset bellow do show that more data indicates for validation accuracy to be displayed in analog style. The result is similar to Test case 2 while it differentiates from Test case 3. The only downside that this test case represents is lower accuracy. Compared to other test cases, this one if the lowest but still in the 70\% range of validation accuracy.
\newline
Due to increased sample size, we can see that in initial dataset, the plotted curve does look more analog. If we compare that to Test case 3, we can see that due to smaller data size the curve is much more digital. This gives us conformation that data augmentation has worked for this case.
\begin{table}[!ht]
  \centering
    \begin{tabular}{ |m{15em}|m{17em}| } 
     \hline
        Variable names & Variable values \\ 
     \hline
        num{\_}epochs & 80 \\ 
     \hline
        batch{\_}size & 50 \\
     \hline
        feature{\_}extractor{\_}num{\_}classes & 3 \\
     \hline
        feature{\_}composer{\_}num{\_}classes & 6 \\
     \hline
        folds & 20 \\
     \hline
        feature{\_}extractor{\_}lr & 0.0001 \\
     \hline
        feature{\_}composer{\_}lr & 0.001 \\
     \hline
        use{\_}cuda & True \\
     \hline
        k & 2 \\
     \hline
        modelType & ResNet18 \\
     \hline
        momentumValue & 0.99 \\
     \hline
        dropoutValue & 0.0 \\
     \hline
    \end{tabular}
\caption{Test case 4 variables}
\end{table}

%The graphs bellow are showing the plotted results:
%\newline
%(GRAPH A)
\begin{figure}[H]
    \centerline{\includegraphics[scale=0.6]{img/graph-test4-composed.png}}
    \caption{Test case 4 training and validation accuracy - composed dataset}
\end{figure}
%\newline
%(GRAPH B)
\begin{figure}[H]
    \centerline{\includegraphics[scale=0.6]{img/graph-test4-initial.png}}
    \caption{Test case 4 training and validation accuracy - initial dataset}
\end{figure}
%\newline
%From this information, we can conclude:

\newpage
After executing fourth test case, we can see results:
%\newline
\begin{table}[!ht]
  \centering
    \begin{tabular}{ |m{14em}|m{9em}|m{9em}| } 
     \hline
        Testing names & Composed results & Initial results \\ 
     \hline
        Accuracy & 0.23 & 0.18 \\
     \hline
        Micro Precision & 0.23 & 0.18 \\
     \hline
        Micro Recall & 0.23 & 0.18 \\
     \hline
        Micro F1 score & 0.23 & 0.18 \\
     \hline
        Macro Precision & 0.04 & 0.03 \\
     \hline
        Macro Recall & 0.17 & 0.17 \\
     \hline
        Macro F1 score & 0.06 & 0.05 \\
     \hline
        Weighted Precision & 0.05 & 0.03 \\
     \hline
        Weighted Recall & 0.23 & 0.18 \\
     \hline
        Weighted F1 score & 0.09 & 0.05 \\
     \hline
    \end{tabular}
\caption{Test case 4 results}
\end{table}

%\newline
\begin{figure}[H]
    \centerline{\includegraphics[scale=0.6]{img/res-test4-composed.png}}
    \caption{Test case 4 confusion matrix results - composed dataset}
\end{figure}
%\newline
\begin{figure}[H]
    \centerline{\includegraphics[scale=0.6]{img/res-test4-initial.png}}
    \caption{Test case 4 confusion matrix results - initial dataset}
\end{figure}

%From this information, we can conclude: