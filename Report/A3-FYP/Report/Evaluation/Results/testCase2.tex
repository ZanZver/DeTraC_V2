\subsubsection{Test case 2}
Second test is showing results of much more intensive testing. The goal of this test is to get deeper information by testing the model a bit longer with heavier weights.
\newline
This test produced interesting results for comparison. First of all, if we have a look at training and validation accuracy graph of composed dataset, we can see that validation accuracy seems to be lower than expected but median of the plotted curve is a lot more centralised. If we compare that to initial dataset graph, we can see that validation accuracy is lower but plotted curve seem to have a bit more offset from its median in comparison to composed dataset.
\begin{table}[!ht]
  \centering
    \begin{tabular}{ |m{15em}|m{17em}| } 
     \hline
        Variable names & Variable values \\ 
     \hline
        num{\_}epochs & 100 \\ 
     \hline
        batch{\_}size & 30 \\
     \hline
        feature{\_}extractor{\_}num{\_}classes & 3 \\
     \hline
        feature{\_}composer{\_}num{\_}classes & 6 \\
     \hline
        folds & 20 \\
     \hline
        feature{\_}extractor{\_}lr & 0.0001 \\
     \hline
        feature{\_}composer{\_}lr & 0.0001 \\
     \hline
        use{\_}cuda & True \\
     \hline
        k & 2 \\
     \hline
        modelType & ResNet18 \\
     \hline
        momentumValue & 0.99 \\
     \hline
        dropoutValue & 0.2 \\
     \hline
    \end{tabular}
\caption{Test case 2 variables}
\end{table}

%The graphs bellow are showing the plotted results:
%\newline
%(GRAPH A)
\begin{figure}[H]
    \centerline{\includegraphics[scale=0.6]{img/graph-test2-composed.png}}
    \caption{Test case 2 training and validation accuracy - composed dataset}
\end{figure}
%\newline
%(GRAPH B)
\begin{figure}[H]
    \centerline{\includegraphics[scale=0.6]{img/graph-test2-initial.png}}
    \caption{Test case 2 training and validation accuracy - initial dataset}
\end{figure}
%\newline
%From this information, we can conclude:

%\newpage
After executing second test case, we can see results:
%\newline
\begin{table}[!ht]
  \centering
    \begin{tabular}{ |m{14em}|m{9em}|m{9em}| } 
     \hline
        Testing names & Composed results & Initial results \\ 
     \hline
        Accuracy & 0.08 & 0.16 \\
     \hline
        Micro Precision & 0.08 & 0.16 \\
     \hline
        Micro Recall & 0.08 & 0.16 \\
     \hline
        Micro F1 score & 0.08 & 0.16 \\
     \hline
        Macro Precision & 0.01 & 0.03 \\
     \hline
        Macro Recall & 0.17 & 0.17 \\
     \hline
        Macro F1 score & 0.02 & 0.05 \\
     \hline
        Weighted Precision & 0.01 & 0.03 \\
     \hline
        Weighted Recall & 0.08 & 0.16 \\
     \hline
        Weighted F1 score & 0.01 & 0.04 \\
     \hline
    \end{tabular}
\caption{Test case 2 results}
\end{table}

%\newline
\begin{figure}[H]
    \centerline{\includegraphics[scale=0.6]{img/res-test2-composed.png}}
    \caption{Test case 2 confusion matrix results - composed dataset}
\end{figure}
%\newline
\begin{figure}[H]
    \centerline{\includegraphics[scale=0.6]{img/res-test2-initial.png}}
    \caption{Test case 2 confusion matrix results - initial dataset}
\end{figure}

%From this information, we can conclude: