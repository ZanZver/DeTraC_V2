\subsection{Themes}
As name suggests, DeTraC is made of:
\begin{itemize}
  \item Decomposition,
  \item Transfer learning,
  \item Composition.
\end{itemize}

Due to that, transfer learning is one of the first themes on the list. In the subject of technology, transfer learning has the same principle as in real world. We would train an algorithm on one dataset, and then transfer the knowledge of it on another one.
Second topic is neural networks. To have working algorithm, neural network is going to be used to train and predict an output. Neural networks do create a mesh or simply network of neurons on which computer decides what to do. The only downside of them is that they sometimes cannot be as flexible.
\newline
Since the goal of DeTraC is to process images, we would be focusing on Convolutional neural networks (CNN). This opens up the third topic of discussion. As mentioned above, neural networks do have some downsides, but image comparison is one of the strong points. But CNN is improved point of it, therefore we would increase our accuracy / efficiency of the algorithm.
\newline
Forth theme is deep learning. If we are using neural networks (in general) we could focus on a bit more with deep learning. Deep learning is a branch of neural networks and what it does is teaches itself by example as humans do. Example from industry would be autonomous vehicles which are able to recognise traffic signs.
\newline
The final theme is clustering. The data (images in our case) that is going to be used by algorithm needs to be organised to some extent. With this, neural network can be guided to train itself to recognise correct patterns. Example from real world would be list of animals. If we would like to teach the difference between a cat and a dog to the kid, we would show them different images of each and tell them what is what. Classification uses the same principle, to achieve the same result.
