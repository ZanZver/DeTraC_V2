\subsubsection{VGG16}
Source: \parencite{qassim2018compressed}
\newline
Article title: Compressed residual-VGG16 CNN model for big data places image recognition \newline
Tags: [VGG16], [CNN], [CL], [NN]
\newline
Description:
\newline
Focus of this article is how CNN and VGG16 can increase speed and decrease the size of NN. This can help to form a better understanding for us on how CNN and VGG16 do help our algorithm underneath since we are going to be using CNN in the implementation and VGG16 was a good implementation candidate.
\newline

Source: \parencite{liu2017weld}
\newline
Article title: Weld Defect Images Classification with VGG16-Based Neural Network 
\newline
Tags: [VGG16], [NN], [CL]
\newline
Description:
\newline
In our example, we are using x-ray images for classification. This article uses similar structure in terms of technology, therefore in the implementation stage (or beforehand) we could take advantage of that and see if there are any improvements we can do based on the article.
\newline

Source: \parencite{rezende2018malicious}
\newline
Article title: Malicious Software Classification Using VGG16 Deep Neural Network’s Bottleneck Features
\newline
Tags: [VGG16], [NN], [CL]
\newline
Description:
\newline
Classification is one of the technologies that is going to be implemented. This article is explaining how they are using classification with the help of VGG16 for classifying malicious software. In our case, this article is interesting in terms of classification explanation with connection to VGG16.
\newline

Source: \parencite{zhao2018synthetic}
\newline
Article title: Synthetic Medical Images Using F AND BGAN for Improved Lung Nodules Classification by Multi-Scale VGG16
\newline
Tags: [VGG16], [NN], [CL]
\newline
Description:
\newline
Classification of cancer images with VGG16 – we are doing similar Focus of this article is image classification with VGG16. This project is taking the similar functionality approach to ours, therefore we can learn from it. Do note that the article is doing classification based on cancer images, but in our case we could change the dataset to the same one as in the article.
\newline

Source: \parencite{arsa2019vgg16}
\newline
Article title: VGG16 in Batik Classification based on Random Forest 
\newline
Tags: [VGG16], [NN], [CL], [DL]
\newline
Description:
\newline
Corelation of this project with ours is small, but they are exposing an interesting way of handling unique patterns with deep learning, clustering and VGG16. Due to the combination of technologies used, it can be an interesting topic to go by.
\newline

Source: \parencite{hridayami2019fish}
\newline
Article title: Fish Species Recognition Using VGG16 Deep Convolutional Neural Network
\newline
Tags: [VGG16], [NN], [CL], [CNN]
\newline
Description:
\newline
The article is tackling the best approach of image recognition on fish based on colour (RGB). They do use different colour sectors and that is important for us since there is some colour in images we are using and if this can be enhanced, this article can help us with it.
\newline

Source: \parencite{liu2019improved}
\newline
Article title: Improved Kiwifruit Detection Using Pre-Trained VGG16 With RGB and NIR Information Fusion
\newline
Tags: [VGG16], [NN], [CNN], [CL]
\newline
Description:
\newline
This article is a bit of an extinction on colour based on previous one. The main difference is that this article is using method RGB-D (Red Green Blue-depth). In the article they are using CNN and VGG16 with RGB-D to improve fruit detection. Might not be as useful for us since we are not dealing with much colour in the dataset.
\newline

Source: \parencite{krishnaswamy2020disease}
\newline
Article title: Disease Classification in Eggplant Using Pre-trained VGG16 and MSVM 
\newline
Tags: [VGG16], [NN], [CL]
\newline
Description:
\newline
In this study, they are trying to use multi class support vector machines (MSVM) and VGG16 for disease detection. In our project, we are trying to detect disease as well, therefore this can be helpful.
\newline

Source: \parencite{theckedath2020detecting}
\newline
Article title: Detecting Affect States Using VGG16, ResNet50 and SE-ResNet50 
\newline
Networks Tags: [VGG16], [ResNet] ,[NN], [CNN], [CL]
\newline
Description:
\newline
This article is comparing different networks, VGG16 and ResNet which are our top candidates are in the list. Due to that, this article could help to guide us to which network is the best for our use.
\newline

Source: \parencite{yang2021novel}
\newline
Article title: A novel method for peanut variety identification and classification by Improved VGG16
\newline
Tags: [VGG16], [NN], [CL], [CNN], [ResNet]
\newline
Description:
\newline
This article is describing the new and improved ways VGG16 can detect images. It is compared to ResNet, therefore it can help to form a better decision for the final algorithm of chaise. If VGG16 is the better algorithm, this article can help us to improve the efficiency of it.