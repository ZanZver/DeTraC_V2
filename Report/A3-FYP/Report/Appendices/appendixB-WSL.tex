\subsection{Appendix B: WSL}
WSL stands for Windows Subsystem for Linux. The role WSL takes is, it uses native Windows functions and it passes it to Linux directly. This is different than virtual machine (VM), since VMs are isolated from host (or base) operating system, therefore requiring a lot more resources. 
With WSL, developers can develop applications/programs for Linux in Windows. One great example is multi compatibility of the programs. If program is required to run on Windows and Linux, developer can develop for both operating systems at the same time and test it simultaneously \parencite{web:AboutWSL}.