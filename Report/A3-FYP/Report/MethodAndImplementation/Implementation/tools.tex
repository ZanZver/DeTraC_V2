\subsubsection{Tools}
Listed bellow, we can see two tables describing what tools have been used in this project. This project was done on 100\% open source tools to reduce the cost. 
\newline
Do note that we are using the server from BCU that has Microsoft Windows operating system installed. This was not something we could change, so to work around that WSL (Appendix B: WSL) was added with Ubuntu OS on it. With that in mind, this project can be reproducible without Windows. 
\newline
Third table has server description listed. Server is running as a VM in the host OS.
\begin{table}[ht]
  \centering
    \begin{tabular}{ |m{12em}|m{20em}| } 
     \hline
        Tool name & Tool description \\ 
     \hline
        Server & Windows server, rented by BCU for computational power. \\ 
     \hline
        WSL - Ubuntu & For executing code in Ubuntu terminal. \\
     \hline
        Python & Programming language used for this project. \\
     \hline
        Pip/Conda & Pythons package manager tools.\\
     \hline
    \end{tabular}
\caption{Main project tools}
\end{table}

\begin{table}[ht]
  \centering
    \begin{tabular}{ |m{12em}|m{20em}| } 
     \hline
     Python package name & Package description \\ 
     \hline
        cv2 & image augmentation (eg: resize image) \\
     \hline
        datetime & get the current date \\
     \hline
        keras & for data augmentation \\
     \hline
        loguru & for better logging \\
     \hline
        matplotlib & allows us to draw graphs \\
     \hline
        numpy & better mathematical number handling (ex: numpy array) \\
     \hline
        os & allows us to access files/folders with Python \\
     \hline
        pytorch & AI and ML libraries \\
     \hline
        random & for random image shuffle \\
     \hline
        shutil & for creating new files/folders with Python \\
     \hline
        sklearn & for data science libraries (ex: KMeans) \\
     \hline
        tensorflow & AI and ML libraries \\
     \hline
        tqdm & displays progress bar in terminal \\
     \hline
    \end{tabular}
\caption{Main Python packages}
\end{table}

\begin{table}[ht]
  \centering
    \begin{tabular}{ |m{12em}|m{20em}| } 
     \hline
     VM component &  VM component specification \\ 
     \hline
        OS & Windows 10 \\
     \hline
        CPU & 16 cores \\
     \hline
        RAM & 64GB \\
     \hline
        GPU & 48GB \\
     \hline
        Storage & 700GB \\
     \hline
        Privileges & Administrator \\
     \hline
    \end{tabular}
\caption{Server specification}
\end{table}