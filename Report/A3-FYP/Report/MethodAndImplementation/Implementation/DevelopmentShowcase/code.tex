\paragraph{Code}

This folder is home to all of the Python files. Main Python file (main.py) is responsible for executing the correct subprograms when needed. This modular approach is  better for the long run if we want to change any functionality. Another bonus of this approach is readily. It is simple to read the code that is organised across multiple files in compare to having one big file.
\newline
In the table bellow, all of the programs that are used are listed and their functionality is explained. There is also a tree structure provided for visualisation of the "Code" folder.
\newline
Note: requirements.txt file contains versions of libraries used in the program.
\begin{table}[!ht]
  \centering
    \begin{tabular}{ |m{13em}|m{19em}| } 
     \hline
        Program name & Program description \\ 
     \hline
        confusionMatrix.py & Creates new confusion matrix \\
     \hline
        createNewInitaialDataset.py & Creates new folder initial{\_}dataset{\_}v2 and fills it with original data \\
     \hline
        dataAugmentation.py & If needed, data can be augmented by adding new "synthetic" images \\
     \hline
        decomposition.py & Decomposes the images into the correct folder \\
     \hline
        featureExtraction.py & Extracts features and save them locally \\
     \hline
        imagePreprocessing.py & Preprocesses images to fit the size and correct colour \\
     \hline
        kFold.py & Does k fold cross validation \\
     \hline
        main.py & Body of the code is here \\
     \hline
        model.py & Model is build in this section \\
     \hline
        modelPrediction.py & Predictions are executed here \\
     \hline
        multiclassConfusionMatrix.py & Helper class for confusion matrix \\
     \hline
        trainFeatureComposer.py & Main body for feature composer - everything is build here \\
     \hline
        trainFeatureExtractor.py & Main body for feature extractor - everything is build here \\
     \hline
    \end{tabular}
\caption{Programs described}
\end{table}

\begin{figure}[!ht]
\dirtree{%
.1 DeTraC{\_}V2.
% Code folder
.2 Code.
.3 confusionMatrix.py.
.3 createNewInitaialDataset.py.
.3 dataAugmentation.py.
.3 decomposition.py.
.3 featureExtraction.py.
.3 imagePreprocessing.py.
.3 kFold.py.
.3 main.py.
.3 model.py.
.3 modelPrediction.py.
.3 multiclassConfusionMatrix.py.
.3 trainFeatureComposer.py.
.3 trainFeatureExtractor.py.
.3 requirements.txt.
}
\caption{Example of tree folder structure of Code folder}
\end{figure}